\documentclass[12pt]{article}
\usepackage{geometry}                % See geometry.pdf to learn the layout options. There are lots.
\geometry{letterpaper}                   % ... or a4paper or a5paper or ... 
%\geometry{landscape}                % Activate for for rotated page geometry
\usepackage[parfill]{parskip}    % Activate to begin paragraphs with an empty line rather than an indent
\usepackage{daves,fancyhdr,natbib,graphicx,dcolumn,amsmath,lastpage,url}
\usepackage{amsmath,amssymb,epstopdf,longtable}
\usepackage{paralist} 
\usepackage[final]{pdfpages}
\DeclareGraphicsRule{.tif}{png}{.png}{`convert #1 `dirname #1`/`basename #1 .tif`.png}
\pagestyle{fancy}
\lhead{CE 3372 -- Water Systems Design}
\rhead{FALL 2017}
%\rhead{FALL 2016}
%\rhead{SPRING 2016}
%\rhead{FALL 2011}
%\rhead{SPRING 2012}
%\rhead{FALL 2012}
%\rhead{FALL 2015}
%\rhead{FALL 2010}
%\lfoot{EXERCISE 1 -- REVISION 1}
%\lfoot{EXERCISE 1 -- REVISION 2}
%\lfoot{EXERCISE 1 -- REVISION 3}
%\lfoot{EXERCISE 1 -- DUE 26 JAN 2012}
%\lfoot{EXERCISE 1 -- DUE 4 SEP 2012}
\lfoot{EXERCISE 5}
\cfoot{}
\rfoot{Page \thepage\ of \pageref{LastPage}}
\renewcommand\headrulewidth{0pt}
\newcommand\tab[1][1cm]{\hspace*{#1}}


\begin{document}
\begin{center}
\textbf{MEMORANDUM}
%{\textbf{{ CE 3372 -- Water Systems Design} \\ {Exercise Set 2}}}
\end{center}
\begingroup
\begin{tabular}{p{1in} p{5in}}
To: & P. N Guin \\ ~\\
From: & P. Olar Bear \\ ~\\
Date: & 04JAN2024 \\ ~\\
Subject: & CE 3372 -- Water Systems Design, Exercise Set 5. ~\\

\end{tabular}
\endgroup
\section*{\small{Purpose}}  
This memorandum presents solutions to several relevant hydraulics problems involving head loss in pipes.  
\section*{\small{Discussion}}
The three problems apply the Hazen-Williams equation, Jain Equation for discharge, and Jain Equation for diameter.  
The results for each problem are presented below; with the actual analysis presented in the attachment.
\subsection*{\small{Problem 1}}
Problem 1 asks for the conversion of the Hazen-Williams formula into discharge form.  The result is
\begin{equation}
h_f = \frac{7.883}{(1.318)^{1.852}} \times \frac{Q}{C_h}^{1.852} \times \frac{L}{D^{4.8707}}
\end{equation}
The algebra is shown on the attachment pages 1-2.

An estimate of $C_h$ for epoxy-coated steel is 145, from Table 6.1 in  \url{http:\\ncrpb.nic.in}.  
Using the estimate and the head loss equation above the estimated head loss for the conditions provided is $69.6$~feet.
The value was checked using a spreadsheet calculator built for Hazen-Williams head loss models (supplied on the class server).

\subsection*{\small{Problem 2}}
Problem 2 presents the Swamee--Jain equation for discharge given head loss, length, diameter, roughness height, and viscosity, and requests we find values for viscosity at $50^o$~F and roughness height for iron pipe, determine the equivalent height of a column of water that would produce a pressure of $420$~psi and finally determine the volumetric flow rate in a pipeline given elevation and pressure changes (and diameter and material).

The found viscosity values is $\nu = 1.41 \times 10^{-5}$~feet$^2$/second.  
The reference used is the water properties database located at \url{http:\\cleveland1.ddns.net\mytoolbox-server}

The found roughness height is $0.0002$~inches. The reference used is \url{http:\\engineersedge.com\fluid_flow\pipe-roughness.htm}

The equivalent height of a column of water at $20$~psi is $46.154$~feet.  The hydraulic analysis for this equivalent height is shown in the attachments (pg 6).

The computed flow rate in the pipeline is $9.7$~cfs.   
The values are verified by both the online calculator at \url{http:\\cleveland1.ddns.net\mytoolbox-server} and the spreadsheet calculator used in lecture 5.

\subsection*{\small{Problem 3}}
Problem 3 presents the Swamee--Jain equation for diameter given head loss, length, discharge, roughness height, and viscosity. 
The viscosity is $\nu = 1.22 \times 10^{-5}$~feet$^2$/second.
The  roughness height selected is $0.0002$~inches.

The pipeline connects two reservoirs, $2$~miles apart, that have a total head difference of $20$~feet.
Application of the Swamee--Jain equation for diameter produced an estimate (by-hand) of $D = 1.78$~feet.  
The value was verified using the online calculator at \url{http:\\cleveland1.ddns.net\mytoolbox-server} 


\subsection*{\small{Concluding Remarks}}
These problems required analysis and application of principles and tools presented in Lecture 5, Head Loss Models.   
The problems are all worked by-hand,  and verified using online and user-built spreadsheets.  


\begin{tabular}{p{6in}}
\hline
~\\
\end{tabular}

Sincerely, \\
P. Olar Bear \\
Icehaus GmBH \\
\\Attachment(s):\\
(1) By-hand analysis for Problem 1 -- 3; Including printouts of indicated references, on-line calculators, and user-written spreadsheet programs.

\includepdf[pages=1-17]{./Scan3.pdf}
 


\end{document}  